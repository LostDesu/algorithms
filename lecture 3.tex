\documentclass[a4paper,12pt]{article} 
\usepackage[T2A]{fontenc}			
\usepackage[utf8]{inputenc}			
\usepackage[english,russian]{babel}	
\usepackage{amsmath,amsfonts,amssymb,amsthm,mathtools} 
\usepackage{wasysym}
\usepackage{amsmath}
\everymath{\displaystyle}

\author{конспект от TheLostDesu}
\title{}
\date{\today}


\begin{document}
\maketitle
\section{}
Машина Тьюринга называется самоприменимой, если она останавливается, когда в качестве входного слова для нее используется описание самой машины. Проблема самоприменимости - вопрос о существовании алгоритма, опрееляющяя самопринемимость любой машины T. Доказывается неразрешимости этой проблемы просто - в доказательстве неразрешимости проблемы останова машина применялась сама к себе. 
\section{Алгоритм Маркова}
Нормальный алгоритм Маркова задается конечной последовательностью подстановок. Такт работы алгоритма состоит в поиске подстановки применимой к обрабатываемому слова.\\
Поиск подстановки идет с первой подстановки в последовательности.\\
Если ни одна подстановка не применима - алгоритм завершается \\
Первая найденая подстановка применяется к первому вхождению в строке.\\
Конец обозначается за $\mapsto$\footnote{mapsto в латехе}\\
Все что можно выполнить в Машине Тьюринга можно решить алгоритмом Маркова.
Можно доказать эквивалентность алгоритма Маркова и машины Тьюринга конструктивным путем: Можно построить универсальную МТ, которая могла бы интерпретировать алгоритм Маркова и наоборот. \\
Существуют и другие способы формально описывать алгоритмы, и для всех таких систем можно доказать их эквивалентность МТ.
\section{Простейший компьютер}
Простейший компьютер состоит из: процессора, шины, основной памяти и внешней памяти. Обычно процессор представляют в виде маленькой машины умеющей изменять нечто из основной памяти. При этом все обычно записывается в двоичной СИ. Также, для скорости часто используют <<регистры>>. Это - маленькая память для хранения дополнительных данных. В процессорах может использовать простые арифметические операции, брать данные из памяти, класть данные в память.

Для того, чтобы удобно пользоваться компьютером нужен более удобный для человека язык обращения с компьютером. Для того чтобы он мог существовать нужен инструмент, позволяющий переводить язык из удобного для человека в удобный для компьютера. А так же нужно средство отладки, ведь неудобно дебажить то, что написанно на непонятном для человека языке компьютера.

\section{Си}
В 1973 году выходит первая версия Си. Он разрабатывался, как язык универсальной системе UNIX.\\
В 1989 появляется самый первый стандарт - C89.
В 1999 появляется стандарт C99. Он позволял перемещать программы между системами, приводя язык к одному и тому же формату везде.\\
В 2011 появился стандарт C11, который был исправлен в 2018.

\subsection{Характеристики Си}
1. Императивный язык\footnote{Си никак не ограничивает программиста. Даже если задачу делать сложно, или задача имеют большую вероятность ошибки - Си позволит выполнить эту задачу}\\
2. Удобный синтаксис\\
3. Позволяет оперировать <<машинными>> понятиями\\
4. Переносимость кода\\
5. Хорошие библиотеки\\
6. Удобные оптимизирующие компиляторы.\\
\subsection{Первая программа на Си}
$\#$include <stdio.h>\footnote{Директива(обычно начинается с $\#$, позволяющяя подключить системную библиотеку. Она позволит пользоватся функциями, которые в ней прописаны. Например printf-ом}\\
int main(void)\footnote{Создание функции main. В си это стартовая функция}\\
$\{$

printf("Hello world $\backslash$n"); \footnote{Специальный символ бэкслэш. Позволяет }

return 0;\footnote{То, что возвращает системная функция. По этому возврату система понимает, успешно ли выполнилась функция.}\\
$\}$
\subsection{Память в Си}
Память и переменные бывает:\\
Регистровой\\
Автоматической - компилятор сам выделяет и очищает память\\
Статической - существует от начала и до конца программы\\
Динамической - пытается быть в регистре, чтобы к ней был максимально быстрый доступ.
\subsection{Типы данных}
char - симольный\\
int - целый\\
float - с плавающей точкой\\
double - двойной точности\\
$\_$Complex \footnote{со стандарта C99, однако после стала необязательной} - комплексный.\\ 
Для большего удобства были введены модификаторы: signed, unsigned, long, short, long long. Их пишут перед названием типа.\\
Если есть хоть один модификатор - слово int можно их убрать. Также можно не писать signed для int.\\
К типу int применимо все модификаторы. \\
К типу char - signed и unsigned.\\
К типу double - только long\footnote{начиная с C99}
\subsection{Представление целых чисел}
Байты в представлении числа идут подряд.\\
Порядок байт не гарантируется.\\
Порядок бит в байте не гарантируется.\\
Отрицательные числа часто представляются в дополнительном коде\footnote{Самый значащий тип - знаковый. положительные значения пишутся как обычно. Отрицательные как $2^n$ + x}. 
\end{document}