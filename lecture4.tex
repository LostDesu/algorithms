\documentclass[a4paper,12pt]{article} 
\usepackage[T2A]{fontenc}			
\usepackage[utf8]{inputenc}			
\usepackage[english,russian]{babel}	
\usepackage{amsmath,amsfonts,amssymb,amsthm,mathtools} 
\usepackage{wasysym}
\usepackage{amsmath}
\everymath{\displaystyle}

\author{конспект от TheLostDesu}
\title{}
\date{\today}


\begin{document}
\maketitle
\section{Типы данных. Размеры типов}
int x $\rightarrow $ sizeoff(x) == sizeoff(int).
Файл limits.h задает минимальные и максимальные значения типов.\\
char = 1 байт\\
short $\geq$ 2\\
int $\geq$ 2\\
long $\geq$ 4\\
long long $\geq$ 8\\
В $inttypes.h$ лежат типы фиксированного размера. Например $uint32_t$ - 32х битный тип.\\
Тип $\_Bool$(C99 значения 0 или 1). Необходимо подключать $stdbool.h$\\
Тип $\_Complex$(C99). Необходимо подключить complex.h. В C11  поддержка комплексных чисел стала необязательной.\\
Хорошие компании стараются не ломать старые типы(обратная совместимость). Поэтому bool называется так странно, так как вряд-ли кто-то пользовался таким названием. bool занимает столько же места, сколько и char.
\section{Область действия переменных}
Область действия переменных(scope) - показывает блок, в котором можно пользоваться переменной.\\
Переменная называется локальной, когда она объявленна в функции. Ей можно пользоватся только в функции.\\
Переменная называется глобальной, когда она объявленна вне функции. Ей можно пользоватся везде. \\
Если объявить локальную переменную, то она <<затенит>> глобальную с тем же названием.\\
Ключевыми словами перед int можно указать тип памяти, в которой будет лежать переменная. \\
При объявлении переменной ее стоит инициализировать. Например int x = 42.
\section{Что можно делать с переменными}
\subsection{Арифметические операции}
a + b - сложение\\
a - b - вычитание\\
a * b - умножение\\
a / b - $целочисленное$\footnote{в целочисленных типах} деление\\
a $\%$ b - остаток от деления 
\subsection{Логические операции}
В C все что не 0 - true, а 0 - false. \\
! - инвертирование.\\
|| - или\\
$\& \&$ - и\\
\subsection{Операции с одной переменной}
a += b - добавить к a b\\
a *= b\\ 
a ++ или ++ a - добавить к a 1\\
a -- или -- a - вычесть из a 1\\
Можно явно привести переменную к типу (type)a. При этом, если тип, к которому приводится переменная уже, чем тип переменной - отрезается самый старший бит.
Если все числа могут привестись к int, то возникает integer promootion. Все,что ниже инта будет превращено в него.
\end{document}